\documentclass{article}
\usepackage[margin=0.6in]{geometry}
\usepackage{amsmath}
\usepackage{amssymb}
\usepackage{textcomp}
\usepackage[official]{eurosym}
\usepackage{hyperref}
\usepackage{longtable}
\usepackage{array}
\usepackage{adjustbox}
\usepackage{enumitem}
\setlength{\parindent}{0pt}
\setlist[itemize]{leftmargin=2em}
\setlist[enumerate]{leftmargin=2.5em}
% Number subsubsections as 1, 2, 3 (no parent prefixes like 0.0.1)
\setcounter{secnumdepth}{3}
\renewcommand\thesubsubsection{\arabic{subsubsection}}
\usepackage{iftex}
\ifPDFTeX
  \usepackage[utf8]{inputenc}
  \usepackage[T1]{fontenc}
  \usepackage{lmodern}
\else
  \usepackage{fontspec}
  \newcommand{\TrySetMono}[1]{\IfFontExistsTF{#1}{\setmonofont{#1}}{}}
  \TrySetMono{Consolas}
  \TrySetMono{DejaVu Sans Mono}
  \TrySetMono{Fira Code}
  \TrySetMono{Courier New}
\fi

\begin{document}

\section{md2tex}

A small, no-frills Markdown $\rightarrow$ LaTeX/PDF converter written in Python. One command takes a \texttt{.md} file and produces \texttt{.tex} and \texttt{.pdf} with sensible defaults. \newline

\subsection{Highlights}

\begin{itemize}
\item One-step pipeline: \texttt{.md → .tex → .pdf} (runs a LaTeX engine for you)
\item Cross‑platform engine detection: \texttt{pdflatex}, \texttt{xelatex}, or \texttt{lualatex} (Windows/Linux/macOS)
\item Engine‑flexible LaTeX preamble (via \texttt{iftex}) so the same \texttt{.tex} compiles on Overleaf and locally with pdfLaTeX/LaTeX or XeLaTeX/LuaLaTeX
\item Unicode‑safe behavior:
\begin{itemize}
\item Preserves raw Unicode exactly inside fenced code blocks (```/\textasciitilde{}\textasciitilde{}\textasciitilde{})
\item Uses a Unicode‑capable engine automatically when needed
\end{itemize}
\item Emoji/sticker removal by Unicode ranges (no per‑emoji lists)
\item Inline formatting: \texttt{**bold**}, \texttt{[links](url)}, and `\texttt{ }inline code\texttt{ }`
\item Math: inline `$...$\texttt{, display }\$\$...\$\$\texttt{, and bracketed display blocks using lines with }[\texttt{ and }]`
\item Auto‑wraps common math used in text (e.g., \texttt{\textbackslash{}alpha}, \texttt{\textbackslash{}int}, \texttt{\textbackslash{}vec\{x\}}, \texttt{x\_1}, \texttt{x\textasciicircum{}2}) into `$...$`
\item Tables: GitHub‑style pipe tables scaled to page width
\item Lists: ordered/unordered, nested by indentation (2 spaces per level)
\item Headings: \texttt{\#} $\rightarrow$ \texttt{\textbackslash{}section}, \texttt{\#\#} $\rightarrow$ \texttt{\textbackslash{}subsection}, \texttt{\#\#\#} $\rightarrow$ \texttt{\textbackslash{}subsubsection}, \texttt{\#\#\#\#} $\rightarrow$ \texttt{\textbackslash{}paragraph}
\begin{itemize}
\item Leading numeric prefixes like \texttt{1. Title} are stripped from heading text
\item Subsubsections numbered as \texttt{1, 2, 3} (no \texttt{0.0.1})
\end{itemize}
\item Horizontal rules: \texttt{---}, \texttt{***}, or \texttt{\_\_\_}
\item Newlines: a single newline in Markdown becomes a visible line break in LaTeX (\texttt{\textbackslash{}newline})
\item Cleanup: removes LaTeX aux files after a successful build (keeps only \texttt{.md}, \texttt{.tex}, \texttt{.pdf}, \texttt{.py})
\end{itemize}

\subsection{Requirements}

\begin{itemize}
\item Python 3.8+
\item A LaTeX distribution:
\begin{itemize}
\item Windows: MiKTeX or TeX Live
\item Linux/macOS: TeX Live
\end{itemize}
\end{itemize}

The script auto‑detects engines via PATH and known install locations and prefers \texttt{xelatex}/\texttt{lualatex} when it detects non‑ASCII inside fenced code blocks; otherwise it uses \texttt{pdflatex}. \newline
The generated \texttt{.tex} includes an engine‑aware preamble: pdfTeX uses \texttt{inputenc} + \texttt{T1} + \texttt{lmodern}, while Xe/LuaLaTeX use \texttt{fontspec}. \newline

\subsection{Usage}

\begin{itemize}
\item Windows (PowerShell):
\end{itemize}

\begin{verbatim}
python md2tex.py your_file.md
\end{verbatim}

\begin{itemize}
\item Linux/macOS:
\end{itemize}

\begin{verbatim}
python3 md2tex.py your_file.md
\end{verbatim}

Outputs: \newline
\begin{itemize}
\item \texttt{your\_file.tex} --- generated LaTeX
\item \texttt{your\_file.pdf} --- compiled PDF (if a LaTeX engine is installed)
\end{itemize}

Tip: If you run the script without a file, or pass "/" or ".", it defaults to \texttt{README.md} in the current directory. For example: \newline

\begin{itemize}
\item Windows (PowerShell):
\end{itemize}

\begin{verbatim}
python md2tex.py
python md2tex.py /
\end{verbatim}

\begin{itemize}
\item Linux/macOS:
\end{itemize}

\begin{verbatim}
python3 md2tex.py
python3 md2tex.py /
\end{verbatim}

\subsection{Markdown support details}

\begin{itemize}
\item Paragraphs/newlines
\begin{itemize}
\item Single newline $\rightarrow$ \texttt{\textbackslash{}\textbackslash{}newline} (forced line break)
\item Blank line $\rightarrow$ paragraph break
\end{itemize}
\item Code
\begin{itemize}
\item Fenced code blocks (``\texttt{/~~~) are emitted as }verbatim` with Unicode preserved
\item Inline code uses \texttt{\textbackslash{}texttt\{...\}} with safe escaping
\end{itemize}
\item Math
\begin{itemize}
\item Inline: `$...$`
\item Display: \texttt{\$\$...\$\$} or bracketed block between lines \texttt{[} and \texttt{]}
\item Literal \texttt{\$\$...\$\$} text is preserved (escaped) in regular paragraphs
\item Auto‑math wrapping: if you accidentally use math commands in text (e.g., \texttt{\textbackslash{}alpha}, \texttt{\textbackslash{}int}, \texttt{x\_1}, \texttt{x\textasciicircum{}2}, \texttt{\textbackslash{}vec\{x\}}), they are wrapped into `$...$` automatically
\end{itemize}
\item Tables
\begin{itemize}
\item Pipe tables with a header and a separator line are supported and auto‑scaled to \texttt{\textbackslash{}textwidth}
\end{itemize}
\item Lists
\begin{itemize}
\item \texttt{-}, \texttt{*} unordered; \texttt{1.} ordered
\item Nesting by 2‑space indentation per level
\end{itemize}
\item Headings
\begin{itemize}
\item \texttt{\#}, \texttt{\#\#}, \texttt{\#\#\#}, \texttt{\#\#\#\#} $\rightarrow$ LaTeX sectioning commands
\item Leading numbering like \texttt{1. Title} in source is removed from the title text
\end{itemize}
\item Emojis/stickers
\begin{itemize}
\item Removed globally by Unicode ranges (flags, pictographs, emoticons, dingbats, skin tones, VS‑16/ZWJ)
\end{itemize}
\end{itemize}

\subsection{Troubleshooting}

\begin{itemize}
\item “No LaTeX engine found”
\begin{itemize}
\item Install MiKTeX (Windows) or TeX Live (Linux/macOS) and ensure binaries are on PATH
\end{itemize}
\item “PDF compilation failed”
\begin{itemize}
\item Check the generated \texttt{.tex} next to your \texttt{.md}
\item Make sure packages like \texttt{amsmath}, \texttt{hyperref}, \texttt{adjustbox} are available in your LaTeX install
\end{itemize}
\item “fontspec only works with Xe/LuaLaTeX”
\begin{itemize}
\item The output \texttt{.tex} avoids loading \texttt{fontspec} on pdfLaTeX/LaTeX via \texttt{iftex}. If you manually edit the preamble, keep \texttt{fontspec} under the Xe/Lua branch only.
\end{itemize}
\item “Unicode in code block breaks with pdflatex”
\begin{itemize}
\item The script prefers \texttt{xelatex}/\texttt{lualatex} when it detects non‑ASCII in code fences; install one of them if missing
\end{itemize}
\item Overleaf notes
\begin{itemize}
\item You can compile the same \texttt{.tex} with pdfLaTeX, XeLaTeX, or LuaLaTeX. If you hit Unicode issues, switch the Overleaf compiler to XeLaTeX or LuaLaTeX.
\end{itemize}
\end{itemize}

\subsection{Known limitations (by design)}

\begin{itemize}
\item Images, blockquotes, and task lists are not implemented (kept intentionally simple)
\item This is not a full Markdown parser; it covers the most common patterns used in notes/technical docs
\end{itemize}

\subsection{Comprehensive sample (README as test)}

This README doubles as the end-to-end test document. You can run the converter directly on it to produce a PDF: \newline

\begin{itemize}
\item Windows (PowerShell):
\end{itemize}

\begin{verbatim}
python md2tex.py README.md
\end{verbatim}

\begin{itemize}
\item Linux/macOS:
\end{itemize}

\begin{verbatim}
python3 md2tex.py README.md
\end{verbatim}

Below is the full “hard cases” sample previously in \texttt{test.md}. \newline

\section{Advanced Mathematical Document}

This document tests all markdown features including special characters, equations, tables, and more. \newline

\subsection{Mathematical Equations}

\subsubsection{Inline Mathematics}

The quadratic formula is $x = \frac{-b \pm \sqrt{b^2 - 4ac}}{2a}$ and Euler's identity is $e^{i\pi} + 1 = 0$. \newline

The area of a circle: $A = \pi r^2$ where $r$ is the radius. \newline

\subsubsection{Block Equations}

\textbf{Partial Differential Equation (Heat Equation):} \newline

\[
\frac{\partial u}{\partial t} = \alpha \frac{\partial^2 u}{\partial x^2}
\]

\textbf{Navier-Stokes Equation:} \newline

\[
\rho \left( \frac{\partial \mathbf{v}}{\partial t} + \mathbf{v} \cdot \nabla \mathbf{v} \right) = -\nabla p + \mu \nabla^2 \mathbf{v} + \mathbf{f}
\]

\textbf{Integral Example:} \newline

\[
\int_{0}^{\infty} e^{-x^2} dx = \frac{\sqrt{\pi}}{2}
\]

\textbf{Double Integral:} \newline

\[
\iint_D f(x,y) \, dA = \int_a^b \int_c^d f(x,y) \, dy \, dx
\]

\textbf{Matrix Example:} \newline

\[
\mathbf{A} = \begin{bmatrix}
a_{11} & a_{12} & a_{13} \\
a_{21} & a_{22} & a_{23} \\
a_{31} & a_{32} & a_{33}
\end{bmatrix}
\]

\textbf{Matrix Multiplication:} \newline

\[
\mathbf{C} = \mathbf{A} \times \mathbf{B} = \begin{pmatrix}
c_{11} & c_{12} \\
c_{21} & c_{22}
\end{pmatrix}
\]

\textbf{Square Root and Fractions:} \newline

\[
\sqrt{x^2 + y^2} = \sqrt{\frac{a}{b} + \frac{c}{d}}
\]

\textbf{Summation and Product:} \newline

\[
\sum_{i=1}^{n} i^2 = \frac{n(n+1)(2n+1)}{6} \quad \text{and} \quad \prod_{i=1}^{n} i = n!
\]

\textbf{Limit Example:} \newline

\[
\lim_{x \to \infty} \frac{1}{x} = 0
\]

\textbf{Taylor Series:} \newline

\[
f(x) = f(a) + f'(a)(x-a) + \frac{f''(a)}{2!}(x-a)^2 + \frac{f'''(a)}{3!}(x-a)^3 + \cdots
\]

\subsection{Special Characters \& Symbols}

\subsubsection{Greek Letters}
$\alpha$ (alpha), $\beta$ (beta), $\gamma$ (gamma), $\delta$ (delta), $\varepsilon$ (epsilon), $\zeta$ (zeta), $\eta$ (eta), $\theta$ (theta), $\lambda$ (lambda), $\mu$ (mu), $\pi$ (pi), $\sigma$ (sigma), $\tau$ (tau), $\varphi$ (phi), $\omega$ (omega) \newline

Uppercase: $\Gamma$ (Gamma), $\Delta$ (Delta), $\Theta$ (Theta), $\Lambda$ (Lambda), $\Xi$ (Xi), $\Pi$ (Pi), $\Sigma$ (Sigma), $\Phi$ (Phi), $\Psi$ (Psi), $\Omega$ (Omega) \newline

\subsubsection{Mathematical Operators}
$\pm$ $\mp$ $\times$ $\div$ $\cdot$ $\sqrt{}$ $\sqrt[3]{}$ $\sqrt[4]{}$ $\infty$ $\propto$ $\approx$ $\neq$ $\equiv$ $\leq$ $\geq$ $\subset$ $\supset$ $\subseteq$ $\supseteq$ $\cap$ $\cup$ $\int$ $\oint$ $\partial$ $\nabla$ $\Delta$ $\prod$ $\sum$ \newline

\subsubsection{Other Symbols}
\textcopyright{} \textregistered{} \texttrademark{} \texteuro{} \pounds{} \textyen{} \textcent{} \S{} \P{} \dag{} \ddag{} \textbullet{} \textperthousand{} $\prime$ $\prime\prime$ $\prime\prime\prime$ $\rightarrow$ $\leftarrow$ $\uparrow$ $\downarrow$ $\leftrightarrow$ $\Rightarrow$ $\Leftarrow$ $\Leftrightarrow$ \newline

\subsection{Complex Tables}

\subsubsection{Table 1: Special Characters in Cells}

\begin{adjustbox}{max width=\textwidth}
\begin{tabular}{|p{0.2\textwidth}|p{0.2\textwidth}|p{0.2\textwidth}|p{0.2\textwidth}|}
\hline
Symbol & Name & LaTeX & Unicode \\
\hline
$\alpha$ & Alpha & \texttt{\textbackslash{}alpha} & U+03B1 \\
\hline
$\beta$ & Beta & \texttt{\textbackslash{}beta} & U+03B2 \\
\hline
$\int$ & Integral & \texttt{\textbackslash{}int} & U+222B \\
\hline
$\sum$ & Sum & \texttt{\textbackslash{}sum} & U+2211 \\
\hline
$\sqrt{}$ & Square Root & \texttt{\textbackslash{}sqrt\{\}} & U+221A \\
\hline
$\infty$ & Infinity & \texttt{\textbackslash{}infty} & U+221E \\
\hline
$\approx$ & Approximately & \texttt{\textbackslash{}approx} & U+2248 \\
\hline
$\neq$ & Not Equal & \texttt{\textbackslash{}neq} & U+2260 \\
\hline
\end{tabular}
\end{adjustbox}


\subsubsection{Table 2: Mathematical Constants}

\begin{adjustbox}{max width=\textwidth}
\begin{tabular}{|p{0.2\textwidth}|p{0.2\textwidth}|p{0.2\textwidth}|p{0.2\textwidth}|}
\hline
Constant & Symbol & Approximate Value & Formula \\
\hline
Pi & $\pi$ & 3.14159265359 & $\pi = \frac{C}{d}$ \\
\hline
Euler's Number & e & 2.71828182846 & $e = \lim_{n \to \infty} (1 + \frac{1}{n})^n$ \\
\hline
Golden Ratio & $\varphi$ & 1.61803398875 & $\phi = \frac{1 + \sqrt{5}}{2}$ \\
\hline
Planck's Constant & h & 6.62607015 $\times$ 10$^-$$^3$$^4$ J$\cdot$s & $E = h\nu$ \\
\hline
\end{tabular}
\end{adjustbox}


\subsubsection{Table 3: Programming Languages \& Operators}

\begin{adjustbox}{max width=\textwidth}
\begin{tabular}{|p{0.14166666666666666\textwidth}|p{0.14166666666666666\textwidth}|p{0.14166666666666666\textwidth}|p{0.14166666666666666\textwidth}|p{0.14166666666666666\textwidth}|p{0.14166666666666666\textwidth}|}
\hline
Language & Addition & Multiplication & Division & Modulo & Power \\
\hline
Python & \texttt{a + b} & \texttt{a * b} & \texttt{a / b} & \texttt{a \% b} & \texttt{a ** b} \\
\hline
C++ & \texttt{a + b} & \texttt{a * b} & \texttt{a / b} & \texttt{a \% b} & \texttt{pow(a, b)} \\
\hline
JavaScript & \texttt{a + b} & \texttt{a * b} & \texttt{a / b} & \texttt{a \% b} & \texttt{a ** b} \\
\hline
Java & \texttt{a + b} & \texttt{a * b} & \texttt{a / b} & \texttt{a \% b} & \texttt{Math.pow(a, b)} \\
\hline
\end{tabular}
\end{adjustbox}


\subsection{Code Blocks}

\subsubsection{Python Code with Special Characters}

\begin{verbatim}
import numpy as np
import matplotlib.pyplot as plt

# Calculate π using Monte Carlo method
def estimate_pi(n_samples=1000000):
    """Estimate π using random points in a square"""
    x = np.random.uniform(-1, 1, n_samples)
    y = np.random.uniform(-1, 1, n_samples)
    inside_circle = (x**2 + y**2) <= 1
    pi_estimate = 4 * np.sum(inside_circle) / n_samples
    return pi_estimate

# Test with special operators: +, -, *, /, %, **, //, &, |, ^, ~, <<, >>
result = (2 ** 3) * (10 // 3) + (15 % 4) - (100 / 7)
print(f"Result: {result:.4f}")

# Unicode in strings
symbols = "α β γ δ ε ζ η θ λ μ π σ τ φ ω"
operators = "± × ÷ √ ∞ ≈ ≠ ≤ ≥ ∫ ∑"
\end{verbatim}

\subsubsection{LaTeX Equation}

\begin{verbatim}
\begin{equation}
\nabla \times \mathbf{E} = -\frac{\partial \mathbf{B}}{\partial t}
\end{equation}

\begin{align}
\nabla \cdot \mathbf{E} &= \frac{\rho}{\epsilon_0} \\
\nabla \cdot \mathbf{B} &= 0 \\
\nabla \times \mathbf{E} &= -\frac{\partial \mathbf{B}}{\partial t} \\
\nabla \times \mathbf{B} &= \mu_0 \mathbf{J} + \mu_0 \epsilon_0 \frac{\part
ial \mathbf{E}}{\partial t}
\end{align}
\end{verbatim}

\subsection{Lists with Special Characters}

\subsubsection{Unordered List}
\begin{itemize}
\item Item with $\alpha$ (alpha) and $\beta$ (beta)
\item Mathematical operators: $\int$ $\sum$ $\prod$ $\sqrt{}$
\item Comparison: $\approx$ $\neq$ $\leq$ $\geq$ $\infty$
\item Arrows: $\rightarrow$ $\leftarrow$ $\uparrow$ $\downarrow$ $\leftrightarrow$
\item Symbols: \textcopyright{} \textregistered{} \texttrademark{} \texteuro{} \pounds{} \textyen{}
\end{itemize}

\subsubsection{Ordered List}
\begin{enumerate}
\item First: Calculate $\int_0^1 x^2 dx = \frac{1}{3}$
\item Second: Evaluate $\sum_{i=1}^{10} i = 55$
\item Third: Solve $\frac{dy}{dx} = 2x$ to get $y = x^2 + C$
\item Fourth: Matrix multiplication $\mathbf{A} \times \mathbf{B}$
\item Fifth: Compute $\lim_{x \to 0} \frac{\sin x}{x} = 1$
\end{enumerate}

\subsubsection{Nested Lists}
\begin{itemize}
\item Top level with $\pi$ $\approx$ 3.14159
\begin{itemize}
\item Nested with $e^{i\pi} + 1 = 0$
\item Another nested: $\sqrt{-1} = i$
\end{itemize}
\item Another top: $\infty$ (infinity)
\begin{itemize}
\item Sub-item: $\lim_{n \to \infty}$
\end{itemize}
\end{itemize}

\subsection{Text Formatting Tests}

\textbf{Bold text} with \_italic text\_ and \texttt{inline code} with special chars: \texttt{α\_β\textasciicircum{}γ} \newline

Regular text with \textbf{bold}, *italic*, and *\textbf{bold italic}* combined. \newline

Text with special characters: @ \# \$ \% \textasciicircum{} \& * ( ) \_ + = \{ \} [ ] | \textbackslash{} : ; " ' < > , . ? / \newline

Escaped characters test: \textbackslash{}\_underscore\textbackslash{}\_ \textbackslash{}*asterisk\textbackslash{}* \textbackslash{}\#hash\textbackslash{}\# \textbackslash{}$dollar\$ \textbackslash{}\%percent\textbackslash{}\% \newline

\subsection{Links and References}

Visit \href{https://www.python.org/}{Python Official} for documentation. \newline

Check out \href{https://numpy.org/}{NumPy} for numerical computing. \newline

Mathematical reference: \href{https://mathworld.wolfram.com/}{Wolfram MathWorld} \newline

\subsection{Advanced Equations Section}

\subsubsection{Schrödinger Equation}

\[
i\hbar\frac{\partial}{\partial t}\Psi(\mathbf{r},t) = \left[-\frac{\hbar^2}{2m}\nabla^2 + V(\mathbf{r},t)\right]\Psi(\mathbf{r},t)
\]

\subsubsection{Maxwell's Equations}

\[
\begin{aligned}
\nabla \cdot \mathbf{E} &= \frac{\rho}{\epsilon_0} \\
\nabla \cdot \mathbf{B} &= 0 \\
\nabla \times \mathbf{E} &= -\frac{\partial \mathbf{B}}{\partial t} \\
\nabla \times \mathbf{B} &= \mu_0\mathbf{J} + \mu_0\epsilon_0\frac{\partial \mathbf{E}}{\partial t}
\end{aligned}
\]

\subsubsection{Einstein Field Equations}

\[
R_{\mu\nu} - \frac{1}{2}Rg_{\mu\nu} + \Lambda g_{\mu\nu} = \frac{8\pi G}{c^4}T_{\mu\nu}
\]

\subsubsection{Fourier Transform}

\[
\hat{f}(\xi) = \int_{-\infty}^{\infty} f(x) e^{-2\pi i x \xi} dx
\]

\subsection{Conclusion}

This document contains: \newline
\begin{itemize}
\item Multiple heading levels (\# \#\# \#\#\# \#\#\#\#)
\item Tables with special characters ($\alpha$ $\beta$ $\gamma$ $\pi$ $\sum$ $\int$)
\item Mathematical equations ($inline$ and \$\$block\$\$)
\item Code blocks with various languages
\item Lists (ordered, unordered, nested)
\item Special symbols (\textcopyright{}\textregistered{}\texttrademark{}\texteuro{}\pounds{}\textyen{})
\item Links and references
\item Text formatting (\textbf{bold}, *italic*, \texttt{code})
\item Greek letters ($\alpha$ $\beta$ $\gamma$ $\delta$ $\varepsilon$ $\zeta$ $\eta$ $\theta$ $\lambda$ $\mu$ $\pi$ $\sigma$ $\tau$ $\varphi$ $\omega$ $\Gamma$ $\Delta$ $\Theta$ $\Lambda$ $\Xi$ $\Pi$ $\Sigma$ $\Phi$ $\Psi$ $\Omega$)
\item Mathematical operators ($\pm$ $\times$ $\div$ $\sqrt{}$ $\infty$ $\approx$ $\neq$ $\leq$ $\geq$ $\int$ $\sum$ $\prod$ $\partial$ $\nabla$)
\item Complex LaTeX equations with matrices, integrals, partial derivatives
\end{itemize}

\subsection{Credits}

\begin{itemize}
\item Author: SDNT8810
\end{itemize}


\end{document}